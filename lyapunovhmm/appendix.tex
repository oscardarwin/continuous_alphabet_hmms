\section{Proofs and Additional Material on \cref{sec:prelims}} \label{app:prelims}

\subsection{Proof of \cref{convergenceLn}}

\convergenceLn*

\begin{proof}
The first part is \cite[Proposition~6]{kief14}.
Towards the second part, the following equalities hold.
\stefan{Define notation $\land$}
\begin{align*}
1 - d(\pi_1, \pi_2) & = \lim_{n \rightarrow \infty} \sum_{w \in \Sigma^n} \min\{\| \pi_1 \Psi(w) \|, \| \pi_2 \Psi(w) \|\} && \text{by \cite[Theorem~ 7]{kief14}}\\
& = \lim_{n \rightarrow \infty} \sum_{w \in \Sigma^n} \min\{L_n(w), 1 \}\| \pi_2 \Psi(w)\| \\
& = \lim_{n \rightarrow \infty} \EE_{\pi_2}\big[\min\{ L_n, 1\}  \big] \\
& = \EE_{\pi_2} \big[ \lim_{n \rightarrow \infty}  \min\{L_n, 1\}  \big] && \text{as } 0 \leq \min\{L_n(w), 1\} \leq 1.
\end{align*}
Then, $\lim_{n \rightarrow \infty}  \min\{L_n, 1\} = 0 \iff \lim_{n \rightarrow \infty}  L_n = 0$.
\end{proof}

\subsection{Details on \cref{sleepcycles}} \label{app:sleepcycles}

In \cite{rockhart13} they derived two embedded Markov chains with the following transition matrices:
\begin{equation*}T_1 = \begin{bmatrix}
0.793 & 0.099 & 0.035 & 0.064 &	0.009 \\
0.078 & 0.769 & 0.006 & 0.144 & 0.003 \\
0.018 & 0.004 & 0.833 & 0.134 & 0.012  \\
0.022 & 0.094 & 0.054 & 0.827 & 0.002 \\
0.011 & 0.005 & 0.035 & 0.005 & 0.945  \\
\end{bmatrix}, T_2 = \begin{bmatrix}
0.641 & 0.109 & 0.031 & 0.040 & 0.015 \\
0.202 & 0.699 & 0.008 & 0.089 & 0.003 \\
0.026 & 0.002 & 0.823 & 0.062 & 0.035 \\
0.123 & 0.189 & 0.114 & 0.808 & 0.016 \\
0.007 & 0.001 & 0.024 & 0.001 & 0.931 \\
\end{bmatrix}.
\end{equation*}

Their HMMs are state-labelled.
For each state $i$, they fit a Dirichlet probability density function (pdf) $f_i$ describing the distribution of observations in $\Delta^3$ emitted at state $i$.
The pdfs of diseased and healthy individuals were so similar that they used the same pdf for both HMMs. % which was estimated from the whole population.
Thus the two HMMs differ only in the transition probabilities.

Since $\Delta^3$ is infinite and in this paper we assume finite observation alphabets, we partition the simplex into the sets
\[
 U_k = \{x \in \Delta^3 \mid f_k(x) \geq \sup_{i} f_i(x)\}
\]
for $k = 1, \dots, 5$.
The set $U_k$ contains the points in $\Delta^3$ most likely to be produced in state~$k$.
We assign a letter $a_k$ for each $U_k$, and define a set of observations $\Sigma = \{a_1, \dots, a_5\}$.
Thus, the probability of producing letter $a_k$ from state $i$ is given as $O_{i,k} = \int_{U_k} f_i(x)\, dx$. We estimated the entries of $O$ using a numerical Monte Carlo technique. We generated 100,000 samples from all 5 Dirichlet distributions in their paper which yielded the estimate
\begin{equation*}
O = \begin{pmatrix}
0.9172&0.0803&0&0.0002&0.0024\\
0.0719&0.8606&0&0.0665&0.0010\\
0&0.0007&0.8546&0.1055&0.0392\\
0.0008&0.0998&0.0663&0.8257&0.0075\\
0.0109&0.0094&0.1046&0.0334&0.8416\\
\end{pmatrix}.
\end{equation*}
Since we consider transition labelled HMMs, we define transition functions $\Psi_1, \Psi_2$ with
\[
 \Psi_m(a_k)_{i,j} = \big( T_m \big)_{i,j} O_{i,k}
\] for $m = 1, 2$.
Let $Q = [10]$. We construct the HMM $(Q, \Sigma, \Psi)$ where
\begin{equation*}
\Psi(a) = \begin{pmatrix}
\Psi_1(a) & 0 \\
0 & \Psi_2(a) \\
\end{pmatrix}
\end{equation*}
for each $a \in \Sigma$.

Let $\pi_1$ and $\pi_2$ be the Dirac distributions on states 1 and 6 respectively. These initial distributions correspond to healthy and diseased individuals started from sleep state 1.


\section{Proofs from \Cref{liexpsubsect}}

\subsection{Proof of \cref{sprtcorrectness}}

\sprtcorrectness*

\begin{proof}
We wish to control the probabilities $\PP_{\pi_2}\big( L_N > B\big)$ and $\PP_{\pi_1}\big( L_N < A\big)$ by choosing suitable values of $A$ and $B$. Let $W_n^1 = \{ w \in \Sigma^\omega \mid  A \leq L_m(w) \leq B ~\forall m < n, L_n < A\}$ then

\begin{align*}
\PP_{\pi_1}\big( L_N < A \big) & = \sum_{n = 1}^\infty \PP_{\pi_1}\big( W_n^1 \big) = \sum_{n = 1}^\infty \sum_{w \in W_n^1} \pi_1 \Psi(w) \1^T = \sum_{n = 1}^\infty \sum_{w \in W_n^1} L_n(w) \pi_2 \Psi(w) \1^T \\
& \leq A \sum_{n = 1}^\infty  \sum_{w \in W_n^1} \pi_2 \Psi(w) \1^T = A \sum_{n = 1}^\infty  \PP_{\pi_2}\big( W_n^1 \big) = A \PP_{\pi_2}\big( L_N < A \big). \\
\end{align*}

Similarly, we may derive $\PP_{\pi_2}\big( L_N > b \big) \geq \frac{1}{b} \PP_{\pi_1}\big( L_N > b\big)$ so it follows that

\begin{align*}
A & \geq \frac{\PP_{\pi_1}\big(  L_N < A\big)}{\PP_{\pi_2}\big(  L_N < A\big)} = \frac{\PP_{\pi_1}\big(  L_N < A\big)}{1 - \PP_{\pi_2}\big(  L_N > B\big)} \\
B & \leq \frac{\PP_{\pi_1}\big(  L_N > B\big)}{\PP_{\pi_2}\big(  L_{N} > B\big)} = \frac{1 - \PP_{\pi_1}\big( L_N < A\big)}{\PP_{\pi_2}\big(  L_N > B\big)}\\
\end{align*}
to guarantee the error bounds $\alpha = \PP_{\pi_1}\big( L_{n^*} < A\big)$ and $\beta = \PP_{\pi_2}\big( L_{n^*} > B\big)$.
\end{proof}

\subsection{Proof of \Cref{asymptoticwald}}


\asymptoticwald*

We will prove \Cref{asymptoticwald} later using results in this section



\probexpzero*
Towards the proof of \Cref{probexp0} we use the following which is Theorem 5 from \cite{kief16}.
\begin{lemma}\label{kief16thm5}
Let $(Q, \Sigma, \Psi)$ be an HMM and let $\pi_1$ and $\pi_2$ be initial distributions. If $\pi_1$ and $\pi_2$ are distinguishable then there is $c > 0$ such that
\begin{equation*}
\PP_{\pi_2}\Big( L_{2|Q|n} \leq 1 \Big) - \PP_{\pi_1}\Big( L_{2|Q|n} \geq 1 ) \Big) \geq 1 - 2\exp \big(-\frac{c^2}{18}n\big).
\end{equation*}
\end{lemma}

\begin{proof}[Proof of \Cref{probexp0}]
By \cref{lem:expoprop} there are a set of bottom SCCs $\mathcal{Z}$ in $\mathcal{B}$. Such that for all $Z \in \mathcal{Z}$ we have $\ell(Z) = \{0\}$. Let $\pi \in [0,1]^Q$ and $r \in Q$ such that $(\supp~\pi, r) \in Z$. Suppose that $\pi$ and $\delta_r$ are distinguishable then by \Cref{kief16thm5} both $\PP_{\delta_r}(L_n^* \geq 1) \leq 2\exp\big( -\frac{c^2}{18}n \big)$ and $\PP_{\pi}(L_n^* \leq 1) \leq 2\exp\big( -\frac{c^2}{18}n \big)$ where $L_n^*$ is the likelihood ratio started from initial distributions $\pi$ and $\delta_r$. Fix $-\frac{c^2}{18} \leq \alpha \leq 0$ and define the event $W_n = \{1 > L_n^* \geq \e^{n\alpha}\}$. Then
\begin{align*}
\PP_{\delta_r}(\lim_{n \rightarrow \infty} \frac1n \ln L_n^* > \alpha) & \leq \PP_{\delta_r}(\liminf_n \{\frac1n \ln L_n^* \geq \alpha\}) \\
& \leq \liminf_n \PP_{\delta_r}(\frac1n \ln L_n^* \geq \alpha) \\
& \leq \liminf_n \PP_{\delta_r}(L_n^* \geq \e^{n\alpha}) \\
& = \liminf_n \Big[ \PP_{\delta_r}(1 > L_n^* \geq \e^{n\alpha}) +  \PP_{\pi_2}(L_n^* \geq 1) \Big]\\
& \leq \liminf_n \Big[ \sum_{w \in W_n} \delta_r \Psi(w) \1^T + 2\exp\big( -\frac{c^2}{18}n \big) \Big]\\
& \leq \liminf_n \Big[ e^{-n\alpha} \sum_{w \in W_n} \pi \Psi(w) \1^T\Big]\\
& \leq \liminf_n \Big[ e^{-n\alpha} \PP_{\pi}(L_n^* < 1)\Big]\\
& \leq \liminf_n \Big[ e^{-n\alpha} 2\exp\big( -\frac{c^2}{18}n \big)\Big]\\
& = 0.
\end{align*}
In particular, $\PP_{\pi_2}(\lim_{n \rightarrow \infty} \frac1n \ln L_n = 0 ) = 0$ which contradicts $\Lambda = \{0\}$. Hence $\pi$ and $\delta_r$ are not distinguishable and so $\PP_{\delta_r}$-almost surely, we have $\lim_{n \rightarrow \infty} L_n^* > 0$. By conditioning on the events $\{a_1 r_1 \cdots a_n r_n \in (\Sigma Q)^* \mid \supp~\pi_1 \Psi(w) = \supp~ \pi, r_n = r\}$ it follows that $E_0 = \{\lim_{n \rightarrow \infty} L_n > 0\}$. We now show the second equality. If $\lim_{n \rightarrow \infty} L_n > 0$ then for $\alpha, \beta$ small enough $L_n$ never crosses the SPRT bounds. Hence, we have $\{\lim_{n \rightarrow \infty} L_n > 0\} \subseteq \bigcup_{\alpha, \beta} \{N_{\alpha, \beta} = \infty\}$. For the converse inclusion, suppose that $N_{\alpha, \beta} = \infty$ for some $\alpha, \beta$ this would contradict $\lim_{n \rightarrow \infty} L_n = 0$ since then $N_{\alpha, \beta}$ would be $\PP_{\pi_2}$-almost surely finite.
\end{proof}

\subsection{Proof of \cref{prop:neginf}}

\propneginf*
\begin{proof}
The right-to-left inclusion is clear.
\newcommand{\pmin}{p_{\mathit{min}}}
Towards the converse, let $\pmin>0$ be the minimum non-zero entry in~$\pi_1$ and all $\Psi(a)$ where $a \in \Sigma$.
Suppose that $L_n>0$ holds for all~$n$.
Then we have for all $n \ge 1$:
\begin{align*}
 \frac1n \ln L_n \ &=\ \frac1n \ln \frac{\| \pi_1 \Psi(w_n) \|}{\| \pi_2 \Psi(w_n) \|} \ \ge\ \frac1n  \ln \| \pi_1 \Psi(w_n) \| \ \ge\ \frac1n \ln \pmin^{n+1} \ = \ \frac{n+1}{n} \ln \pmin \\
                   &\ge\ 2 \ln \pmin\,.
\end{align*}
Thus, $\liexp \ne -\infty$. We have $\sup_{\alpha, \beta} N_{\alpha, \beta} \leq N_\perp$. Also,
\begin{equation*}
\bigcap_{\alpha, \beta} \{L_n \not\in (\frac{\alpha}{1-\beta},\frac{1-\alpha}{\beta})\} = \{L_n = 0\}
\end{equation*}
for all $n \in \NN$ and so $\sup_{\alpha, \beta} N_{\alpha, \beta} = N_\perp$. The final claim follows because $\{N_\perp < \infty\} = E_{-\infty}$.
\stefan{The things about $N_{\alpha, \beta}$ might require justification.}
\end{proof}


\subsection{Proof of \cref{liexpmotivation}} \label{app:liexpmotivation}

Towards the proof of \cref{liexpmotivation} we first show the following lemma.

\begin{lemma}\label{unifintegofN}
The set of random variables $\{\frac{N_{\alpha, \beta}}{-\ln \alpha}\mid 0 < \alpha, \beta \leq \frac12\}$ is uniformly integrable with respect to the measure $\PP_{\pi_2}$; i.e.
\begin{equation*}
\lim_{K \rightarrow \infty} \sup_{\alpha, \beta} \EE_{\pi_2} \left[ -\frac{N_{\alpha, \beta}}{\ln \alpha}\1_{\frac{N_{\alpha, \beta}}{-\ln \alpha} \geq - K} \right] = 0.
\end{equation*}
\end{lemma}
We use the following technical lemma which is Lemma 9 from \cite{kief16}.

\begin{lemma}\label{2016profilethm}
There is a number $c > 0$, computable in polynomial time, such that
\begin{equation*}
\PP_{\pi_2}\Big( L_{2|Q|n} \geq \exp( -\frac{c^2}{36} n ) \Big) \leq 4 \exp\Big( -\frac{c^2}{36} n \Big).
\end{equation*}
\end{lemma}

\begin{proof}[Proof of \Cref{unifintegofN}]
By \Cref{liexpmotivation}, conditioned on $E_\ell$ we have $\lim_{\alpha, \beta \rightarrow 0} \frac{N_{\alpha, \beta}}{\ln \alpha}$ exists $\PP_{\pi_2}$-almost surely. Hence, the convergence is also in $\PP_{\pi_2}$-measure. Therefore, by the Vitali convergence theorem\cite{bog2007} it is sufficient to show that the set of random variables $\{ \frac{N_{\alpha, \beta}}{\ln \alpha} \mid \alpha, \beta \in (0,\frac12) \}$ is uniformly integrable conditioned on $V_k$. In fact, because
\begin{equation}\label{unifintcond}
\lim_{K \rightarrow \infty} \sup_{\alpha, \beta} \EE_{\pi_2} \big[ \frac{N_{\alpha, \beta}}{- \ln \alpha}\1_{\frac{N_{\alpha, \beta}}{-\ln \alpha} \geq - K}\big] \geq \PP_{\pi_2}(E_\ell) \lim_{M \rightarrow \infty} \sup_{\alpha, \beta} \frac{1}{- \ln \alpha}\EE_{\pi_2} \big[ \frac{N_{\alpha, \beta}}{- \ln \alpha}\1_{\frac{N_{\alpha, \beta}}{-\ln \alpha} \geq - K} \mid E_\ell \big],
\end{equation}
It is sufficient to check the uniform integrability condition without conditioning on $V_k$.

For fixed $M \geq \frac{144|Q|}{c^2}$, write $m_\alpha = \floor{\frac{- M \ln \alpha}{2|Q|}}$. It follows that 
\begin{equation*}
\frac{2|Q|m_\alpha}{\ln \alpha} \leq M \text{ and } \alpha \geq \exp{-\frac{c^2}{36} m_\alpha}.
\end{equation*}
Further, $m_\alpha \geq \frac{M \ln 2}{2|Q|} - 1$. The following holds
\begin{align*}
& \quad \EE_{\pi_2} \big[ \frac{N_{\alpha, \beta}}{- \ln \alpha} \1_{\frac{N_{\alpha, \beta}}{-\ln \alpha} \geq M}\big]\\
& = \frac{1}{- \ln \alpha} \sum_{n = 0}^\infty \PP_{\pi_2} \big( N_{\alpha, \beta} \1_{N_{\alpha, \beta} \geq 2|Q|m_\alpha}> n\big) \\
& \leq \frac{2|Q|}{- \ln \alpha}\Big( m_\alpha ~\PP_{\pi_2}(N_{\alpha, \beta} \geq 2|Q|m_\alpha) + \sum_{n = m_\alpha}^\infty \PP_{\pi_2}\big( N_{\alpha, \beta} \geq 2|Q|n\big)\Big)\\
& \leq M \PP_{\pi_2}\big( L_{2|Q|m_{\alpha}} \geq \alpha\big) + \frac{2|Q|}{- \ln \alpha} \sum_{n = m_\alpha}^\infty \PP_{\pi_2}\big( L_{2|Q|n} \geq \alpha\big) \\
& \leq M \PP_{\pi_2}\big( L_{2|Q|m_{\alpha}} \geq \exp{-\frac{c^2}{36} m_\alpha}\big) + \frac{2|Q|}{- \ln \alpha} \sum_{n = m_\alpha}^\infty \PP_{\pi_2}\big( L_{2|Q|n} \geq \exp{-\frac{c^2}{36} n}\big) \\
& \leq 4M \exp{-\frac{c^2}{36}m_\alpha}  + \frac{8|Q|}{- \ln \alpha} \sum_{n = m_\alpha}^\infty \exp{-\frac{c^2}{36} n}\\
& \leq 4M \exp{-\frac{c^2}{36}m_\alpha}  + \frac{8|Q|\exp{-\frac{c^2}{36} m_\alpha}}{- \ln \alpha} \frac{1}{1 - \exp{c^2 / 36}}\\
& \leq 4M \exp{-\frac{c^2}{36} \Big(\frac{M \ln 2}{2|Q|} - 1 \Big)}  + \frac{8|Q|\exp \Big( -\frac{c^2}{36} (\frac{M \ln 2}{2|Q|} - 1 ) \Big) }{\ln 2} \frac{1}{1 - \exp{c^2 / 36}}\\
& \rightarrow 0
\end{align*}
as $K \rightarrow \infty$ where the fourth inequality follows by \Cref{2016profilethm}. Hence, \Cref{unifintcond} must hold.
\end{proof}

\liexpmotivation*

\begin{proof}
Since $\Psimin^n \leq L_n \leq \Psimin^{-n}$ it follows that
\begin{equation*}
N_{\alpha,\beta} \geq \frac{ \min \{ \ln \frac{\alpha}{1 - \beta}, \ln \frac{\beta}{1 - \alpha} \} }{\ln \Psimin}
\end{equation*}
Hence $N_{\alpha, \beta} \rightarrow \infty \ \PP_{\pi_2}$-almost surely as $\alpha, \beta \rightarrow 0$. Consider the case $\ell_k \in (-\infty, 0)$. Let $U_{\alpha,\beta} = \{w \in \Sigma^\omega \mid \ln L_{N_\alpha} \leq \ln \frac{ \alpha}{1 - \beta} \}$. The set $\bigcap_{\alpha, \beta \in (0,1]} U_{\alpha, \beta}^c \subseteq \{L_n \text{ is unbounded}\}$. Hence, $\lim_{\alpha, \beta \rightarrow 0}\1_{U_{\alpha,\beta}} = 1 \ ~\PP_{\pi_2}$-almost surely. Conditioned on $V_k$ it follows that

\begin{equation*}
0 \leq \1_{U_{\alpha, \beta}} \frac{\ln \frac{\alpha}{1 - \beta} - \ln L_{N_{\alpha, \beta}}}{N_\alpha} \leq \1_{U_{\alpha, \beta}}  \frac{\ln L_{N_{\alpha, \beta} - 1}  - \ln L_{N_{\alpha, \beta}}}{N_{\alpha, \beta}}\rightarrow 0 \ \text{ as } \alpha \rightarrow 0.
\end{equation*}
And so
\begin{equation*}
\lim_{\alpha, \beta \rightarrow 0}\frac{\ln \alpha}{N_{\alpha, \beta}} = \lim_{\alpha \rightarrow 0}\frac{\ln \frac{\alpha}{1-\beta}}{N_{\alpha, \beta}} =\lim_{\alpha \rightarrow 0}\frac{\ln L_{N_{\alpha, \beta}}}{N_{\alpha, \beta}} =\ell_k.
\end{equation*}
\end{proof}


\section{Proofs from \cref{sec:qual}} \label{app:qual}

\subsection{Proof of \cref{lem:expoprop}}

\lemexpoprop*
\begin{proof}
\begin{enumerate}
\item
Let $C \subseteq 2^Q \times Q$ be a bottom SCC of~$\B$.
Let $\pi, \pi'$ be distributions on~$Q$ and $q, q' \in Q$ such that $(\supp(\pi),q), (\supp(\pi'),q') \in C$.
Suppose that $\ell \in \Lambda_{\pi,e_{q}}$; i.e.,
\begin{equation} \label{eq:expoprop-ass}
 \PP_{e_{q}}\left(\lim_{n \to \infty} \frac1n \ln \frac{\| \pi \Psi(w_n) \|}{\| e_q \Psi(w_n) \|} = \ell\right) \ = \ x \quad \text{for some } x>0.
\end{equation}
It suffices to show that $\Lambda_{\pi',e_{q'}} = \{\ell\}$, i.e.,
\[
 \PP_{e_{q'}}\left(\lim_{n \to \infty} \frac1n \ln \frac{\| \pi' \Psi(w_n) \|}{\| e_{q'} \Psi(w_n) \|} = \ell\right) \ = \ 1.
\]
By L\'evy's 0-1 law it suffices to show that for all paths $q' a_1 q_1 \cdots a_m q_m$ with $\PP_{e_{q'}}(q' a_1 q_1 \cdots a_m q_m(\Sigma Q)^\omega) > 0$ there is $y>0$ with
\begin{equation} \label{eq:expoprop-goal}
 \PP_{e_{q'}}\left(\lim_{n \to \infty} \frac1n \ln \frac{\| \pi' \Psi(w_n) \|}{\| e_{q'} \Psi(w_n) \|} = \ell \;\middle\vert\; q' a_1 q_1 \cdots a_m q_m (\Sigma Q)^\omega \right) \ \ge \ y.
\end{equation}

Let $u = q' a_1 q_1 \cdots a_m q_m$ be a path with $\PP_{e_{q'}}(u (\Sigma Q)^\omega) > 0$.
Since $C$ is a bottom SCC of~$\B$, we have
\[
\PP_{e_{q'}}\left( \exists\,k \ge m : \supp(\pi' \Psi(a_1 \cdots a_m \cdots a_k)) = \supp(\pi),\ q_k = q \;\middle\vert\; u (\Sigma Q)^\omega) \right)=1\,.
\]
Thus, letting $v = q' a_1 q_1 \cdots a_m q_m \cdots a_k q_k$, with $k \ge m$, be an arbitrary extension of~$u$ with $\PP_{e_{q'}}(v (\Sigma Q)^\omega) > 0$ and $\supp(\pi' \Psi(a_1 \cdots a_m \cdots a_k)) = \supp(\pi)$ and $q_k = q$, we have
\begin{align}
 & \PP_{e_{q'}}\left(\lim_{n \to \infty} \frac1n \ln \frac{\| \pi' \Psi(w_n) \|}{\| e_{q'} \Psi(w_n) \|} = \ell \;\middle\vert\; u (\Sigma Q)^\omega \right) \notag\\
 \ge\ & \PP_{e_{q'}}\left(\lim_{n \to \infty} \frac1n \ln \frac{\| \pi' \Psi(w_n) \|}{\| e_{q'} \Psi(w_n) \|} = \ell \;\middle\vert\; v (\Sigma Q)^\omega \right) \notag\\
 \ge\ & \PP_{e_q}\left( \lim_{n \to \infty} \frac1n \ln \frac{\| (\pi' \Psi(a_1 \cdots a_k)) \Psi(w_n) \|}{ \| (e_{q'} \Psi(a_1 \cdots a_k)) \Psi(w_n) \|} = \ell \right) \notag\\
 \ge\ & \PP_{e_q}\left( \lim_{n \to \infty} \frac1n \ln \frac{\| (\pi' \Psi(a_1 \cdots a_k)) \Psi(w_n) \|}{ \| e_{q} \Psi(w_n) \|} = \ell \ \text{ and } \right. \label{eq:first-event} \\
      & \qquad\ \left. \lim_{n \to \infty} \frac1n \ln \frac{\| (e_{q'} \Psi(a_1 \cdots a_k)) \Psi(w_n) \|}{ \| e_{q} \Psi(w_n) \|} = 0 \right) \label{eq:second-event}
\end{align}
Concerning the event in~\eqref{eq:first-event}, by \eqref{eq:expoprop-ass} and since $\supp(\pi' \Psi(a_1 \cdots a_k)) = \supp(\pi)$, we have
\[
\PP_{e_q}\left( \lim_{n \to \infty} \frac1n \ln \frac{\| (\pi' \Psi(a_1 \cdots a_k)) \Psi(w_n) \|}{ \| e_{q} \Psi(w_n) \|} = \ell \right)
\ \ge\ x\,.
\]
Concerning the event in~\eqref{eq:second-event}, it follows from \cref{convergenceLn}.1 \stefan{better reference?} that
\[
\PP_{e_q}\left( \lim_{n \to \infty} \frac1n \ln \frac{\| (e_{q'} \Psi(a_1 \cdots a_k)) \Psi(w_n) \|}{ \| e_{q} \Psi(w_n) \|} \le 0 \right) \ = \ 1\,.
\]
Further, since $\PP_{e_{q'}}(q' a_1 q_1 \cdots a_k q_k (\Sigma Q)^\omega) > 0$, we have $q \in \supp(e_{q'} \Psi(a_1 \cdots a_k))$ and so
\[
\PP_{e_q}\left( \lim_{n \to \infty} \frac1n \ln \frac{\| (e_{q'} \Psi(a_1 \cdots a_k)) \Psi(w_n) \|}{ \| e_{q} \Psi(w_n) \|} \ge 0 \right) \ = \ 1\,.
\]
Thus, continuing the inequality chain from above, we conclude that
\[ \PP_{e_{q'}}\left(\lim_{n \to \infty} \frac1n \ln \frac{\| \pi' \Psi(w_n) \|}{\| e_{q'} \Psi(w_n) \|} = \ell \;\middle\vert\; u (\Sigma Q)^\omega \right) \ \ge\ x\,,
\]
proving~\eqref{eq:expoprop-goal}, as desired.
\item
Let $(S,q) \in C$ for a bottom SCC~$C$.
If $S = \emptyset$ then we may define $\ell(C) = -\infty$.
Otherwise, let $\pi_S$ denote the uniform distribution on~$S$.
Suppose that $\pi_S$ and~$e_q$ are not distinguishable.
By \cref{cor:probexp0} it follows that $0 \in \Lambda_{\pi_S,e_q}$.
Using part 1 we obtain $\ell(C) = 0$.
Finally, suppose that $\pi_S$ and~$e_q$ are distinguishable.
By \cref{cor:probexp0} it follows that $0 \not\in \Lambda_{\pi_S,e_q}$.
Since $C$ does not contain any states of the form $(\emptyset,q')$, by \cref{prop:neginf} we have $-\infty \not\in \Lambda_{\pi_S,e_q}$.
Using part~1 we obtain $\ell(C) \in (-\infty,0)$.
\item
We define a function~$f$ that maps paths of~$\H$ to paths of~$\B$ as follows.
Set $f(q_0 a_1 q_2 \cdots a_m q_m) := (S_0,q_0) (S_1,q_1) \cdots (S_m,q_m)$ where $S_0 = \supp(\pi_1)$ and $\delta(S_{i-1},a_i) = S_i$ for all $1 \le i \le m$.
The Markov chain~$\B$ is constructed so that for any path $v = (S_0,q_0) (S_1,q_1) \cdots (S_m,q_m)$ we have
\[
 \PP_\iota( v (2^Q \times Q)^\omega) \ = \ \PP_{\pi_2}(f^{-1}(v)(\Sigma Q)^\omega)\,.
\]
Let $C$ be any bottom SCC, and let $\ell = \ell(C)$.
Define the event
\[
 V_C := \{q_0 a_1 q_1 \cdots \in Q(\Sigma Q)^\omega \mid \exists\,m \in \NN : f(q_0 a_1 q_1 \cdots a_m q_m) \text{ ends in~$C$} \}\,.
\]
%Note that the $V_C$ are disjoint.
So we have $\PP_{\iota}(\{\text{visit $C$}\}) = \PP_{\pi_2}(V_C)$, and it suffices to show that $\PP_{\pi_2}(E_\ell \mid V_C) = 1$.
Let $u = q_0 a_1 q_1 \cdots a_m q_m$ be a path with $\PP_{\pi_2}(u (\Sigma Q)^\omega) > 0$ such that $f(u)$ ends in~$C$, say in $(S,q) \in C$, with $q = q_m$.
Thus, $\supp(\pi_1 \Psi(a_1 \cdots a_m)) = S$ and $q \in \supp(\pi_2 \Psi(a_1 \cdots a_m))$.
It suffices to show that $\PP_{\pi_2}(E_\ell \mid u (\Sigma Q)^\omega) = 1$.
We have:
\begin{align}
    & \PP_{\pi_2}(E_\ell \mid u (\Sigma Q)^\omega) \notag \\
 =\ & \PP_{\pi_2}\left(\lim_{n \to \infty} \frac1n \ln \frac{\| \pi_1 \Psi(w_n) \|}{\| \pi_2 \Psi(w_n) \|} = \ell \;\middle\vert\; u (\Sigma Q)^\omega \right) \notag \\
 =\ & \PP_{e_q}\left( \lim_{n \to \infty} \frac1n \ln \frac{\| (\pi_1 \Psi(a_1 \cdots a_m)) \Psi(w_n) \|}{ \| (\pi_2 \Psi(a_1 \cdots a_m)) \Psi(w_n) \|} = \ell \right) \notag \\
 \ge\ & \PP_{e_q}\left( \lim_{n \to \infty} \frac1n \ln \frac{\| (\pi_1 \Psi(a_1 \cdots a_m)) \Psi(w_n) \|}{ \| e_{q} \Psi(w_n) \|} = \ell \ \text{ and } \right. \label{eq:f-e}\\
      & \qquad\ \left. \lim_{n \to \infty} \frac1n \ln \frac{\| (\pi_2 \Psi(a_1 \cdots a_m)) \Psi(w_n) \|}{ \| e_{q} \Psi(w_n) \|} = 0 \right) \label{eq:s-e}
\end{align}
Concerning the event in~\eqref{eq:f-e}, by part~2 and since $\supp(\pi_1 \Psi(a_1 \cdots a_m)) = S$, we have
\[
\PP_{e_q}\left( \lim_{n \to \infty} \frac1n \ln \frac{\| (\pi_1 \Psi(a_1 \cdots a_m)) \Psi(w_n) \|}{ \| e_{q} \Psi(w_n) \|} = \ell \right) \ = \ 1\,.
\]
Concerning the event in~\eqref{eq:s-e}, it follows from \cref{convergenceLn}.1 \stefan{better reference?} that
\[
\PP_{e_q}\left( \lim_{n \to \infty} \frac1n \ln \frac{\| (\pi_2 \Psi(a_1 \cdots a_m)) \Psi(w_n) \|}{ \| e_{q} \Psi(w_n) \|} \le 0 \right) \ = \ 1\,.
\]
Further, since $q \in \supp(\pi_2 \Psi(a_1 \cdots a_m))$, we have
\[
\PP_{e_q}\left( \lim_{n \to \infty} \frac1n \ln \frac{\| (\pi_2 \Psi(a_1 \cdots a_m)) \Psi(w_n) \|}{ \| e_{q} \Psi(w_n) \|} \ge 0 \right) \ = \ 1\,.
\]
Thus, the events in \eqref{eq:f-e} and~\eqref{eq:s-e} occur $\PP_{e_q}$-a.s.
We conclude that $\PP_{\pi_2}(E_\ell \mid u (\Sigma Q)^\omega) = 1$, as desired. \qedhere
\end{enumerate}
\end{proof}

We can finally prove \Cref{asymptoticwald}. We use the fact that conditional expected time of visiting a state in a Markov chain is finite. This follows directly from the main result of \cite{shes13}.

\begin{proof}[Proof of \Cref{asymptoticwald}]
The first point follows by \Cref{unifintegofN} and \Cref{liexpmotivation} using Vitali's convergence theorem. The second point follows from \Cref{probexp0}. Finally, by \Cref{prop:neginf} we have $\sup_{\alpha, \beta}\EE_{\pi_2}[N_{\alpha, \beta} \mid E_{-\infty}] \leq \EE_{\pi_2}[N_\perp \mid E_{-\infty}] < \infty$ since by \Cref{lem:expoprop} $L_n = 0$ if and only if we visit a bottom SCC $C$ such that  $(\emptyset, q) \in C$ for some $q \in Q$. 
\end{proof}

\subsection{Proof of \cref{thm:qual-prob}} \label{app:thm-qual-prob}

Below we refer to the complexity class NC, the subclass of P comprising those problems solvable in polylogarithmic time by a parallel random-access machine using polynomially many processors; see, e.g., \cite[Chapter 15]{Pap94}.
To prove membership in PSPACE in a modular way, we use the following pattern:
\begin{lemma} \label{lem:PSPACE-transducer}
Let $P_1, P_2$ be two problems, where $P_2$ is in NC.
Suppose there is a reduction from $P_1$ to~$P_2$ implemented by a PSPACE transducer, i.e., a Turing machine whose work tape (but not necessarily its output tape) is PSPACE-bounded.
Then $P_1$ is in PSPACE.
\end{lemma}
\begin{proof}
Note that the output of the transducer is (at most) exponential.
Problems in NC can be decided in polylogarithmic space~\cite[Theorem~4]{Borodin77}.
Using standard techniques for composing space-bounded transducers (see, e.g., \cite[Proposition~8.2]{Pap94}), it follows that $P_1$ is in PSPACE.
\end{proof}

Now we prove the following theorem from the main body.
\thmqualprob*
\begin{proof}
\begin{enumerate}
\item
The Markov chain~$\B$ from \cref{lem:expoprop} is exponentially big but can be constructed by a PSPACE transducer, i.e., a Turing machine whose work tape (but not necessarily its output tape) is PSPACE-bounded.
The DAG (directed acyclic graph) structure, including the SCCs, of a graph can be computed in NL, which is included in NC.
Using the pattern of \cref{lem:PSPACE-transducer}, the DAG structure of the Markov chain~$\B$ can be computed in PSPACE.
Thus, there is a PSPACE transducer that computes both~$\B$ and its DAG structure.

For each bottom SCC~$C$, the PSPACE transducer also decides whether $\ell(C) = -\infty$ or $\ell(C) \in (-\infty,0)$ or $\ell(C) = 0$, using \cref{lem:expoprop}.2 and the polynomial-time algorithm for distinguishability from~\cite{kief14}.
Finally, to compute $\PP_{\pi_2}(E_{-\infty})$ and $\PP_{\pi_2}(E_0)$, by \cref{lem:expoprop}.3, it suffices to set up and solve a linear system of equations for computing hitting probabilities in a Markov chain.
This system can also be computed by a PSPACE transducer.
Linear systems of equations can be solved in NC~\cite[Theorem~5]{BorodinGathenHopcroft82}.
Using \cref{lem:PSPACE-transducer} again, we conclude that one can compute $\PP_{\pi_2}(E_{-\infty})$ and $\PP_{\pi_2}(E_0)$ in PSPACE.
\item This part was proved in the main body.
\item The claims concerning $\PP_{\pi_2}(E_{-\infty})$ follow from part~1 and \cref{prop:nontrivial-approx}.
Consider the problem whether $\PP_{\pi_2}(E_{0}) = 1$.
By part~1, it is in PSPACE.
Towards PSPACE-hardness we reduce again from mortality.
Let $(Q,\Sigma,\Phi)$ be an instance of the mortality problem.
Let $Q' := Q \cup \{q_\bot, q_2\}$ for fresh states $q_\bot,q_2$, and let $\Sigma' := \Sigma \cup \{\$\}$ for a fresh letter~$\$$.
Obtain $\Phi'$ from~$\Phi$ by adding, for every $q \in Q'$, a $\$$-labelled transition to~$q_\bot$, and an $a$-labelled loop from $q_2$ to itself for all $a \in \Sigma$.
Construct an HMM $(Q',\Sigma',\Psi)$ so that $\Phi'(a)$ and~$\Psi(a)$ have the same zero pattern for all $a \in \Sigma'$ (e.g., use uniform distributions).
See \cref{fig:PSPACE-hardness}.
\begin{figure}[ht]
\begin{center}
\begin{tikzpicture}[scale=2.5,LMC style]
\node[state] (q0) at (0,0) {$q_0$};
\node[state] (q1) at (1,0) {$q_1$};
\path[->] (q0) edge [loop,out=200,in=160,looseness=10] node[left] {$b$} (q0);
\path[->] (q0) edge node[above] {$a,b$} (q1);
\path[->] (q1) edge [loop,out=20,in=-20,looseness=10] node[right] {$a$} (q1);
\draw[->,line width=3] (1,-0.5) -- (1,-1);

\node[state] (q0') at (0,-1.5) {$q_0$};
\node[state] (q1') at (1,-1.5) {$q_1$};
\node[state] (q2) at (2,-1.5) {$q_2$};
\node[state] (qb) at (1,-2.5) {$q_\bot$};
\path[->] (q0') edge [loop,out=200,in=160,looseness=10] node[left] {$\frac14 b$} (q0');
\path[->] (q0') edge[bend left] node[above] {$\frac14 a$} (q1');
\path[->] (q0') edge[bend right] node[above] {$\frac14 b$} (q1');
\path[->] (q1') edge [loop,out=20,in=-20,looseness=10] node[right] {$\frac12 a$} (q1');
\path[->] (q0') edge node[left,xshift=-2] {$\frac14\$$} (qb);
\path[->] (q1') edge node[left] {$\frac12\$$} (qb);
\path[->] (q2) edge [loop,out=80,in=40,looseness=10] node[right] {$\frac13 a$} (q2);
\path[->] (q2) edge [loop,out=-40,in=-80,looseness=10] node[right] {$\frac13 b$} (q2);
\path[->] (q2) edge node[left] {$\frac13\$$} (qb);
\path[->] (qb) edge [loop,out=20,in=-20,looseness=10] node[right] {$1\$$} (qb);
\end{tikzpicture}
\end{center}
\caption{Illustration of the reduction from mortality to $\PP_{\pi_2}(E_0)<1$.
In this example, $\Phi(a b)$ is the zero matrix.
Accordingly, we have $\PP_{\pi_2}(E_0) < 1$, as $L_2(a b w) = 0$ for all $w \in \Sigma^\omega$.}
\label{fig:PSPACE-hardness}
\end{figure}
Let $\pi_1 \in [0,1]^{Q'}$ be the uniform distribution on~$Q$ (i.e., $(\pi_1)_{q_\bot} = (\pi_1)_{q_2} = 0$), and let $\pi_2$ be the Dirac distribution on~$q_2$.

Suppose $(Q,\Sigma,\Phi)$ is a positive instance of the mortality problem.
Let $v \in \Sigma^*$ such that $\Phi(v)$ is the zero matrix.
Then $L_{|v|}(v w) = 0$ holds for all $w \in \Sigma^\omega$.
It follows that $\PP_{\pi_2}(E_{-\infty})>0$ and so $\PP_{\pi_2}(E_0)<1$.

Conversely, suppose $(Q,\Sigma,\Phi)$ is a negative instance of the mortality problem.
The word produced from~$q_2$ contains $\PP_{e_{q_2}}$-a.s.\ the letter~$\$$, i.e., is of the form $u \$ v$ for $u \in \Sigma^*$ and $v \in (\Sigma \cup \{\$\})^\omega$.
Since $(Q,\Sigma,\Phi)$ is a negative instance, it follows that $\supp(\pi_1 \Psi(u \$)) = \{q_\bot\} = \supp(e_{q_\bot} \Psi(u \$))$.
Thus, $\lim_{n \to \infty} L_n > 0$. 
Hence, $\PP_{\pi_2}(E_0)=1$.
\qedhere
\end{enumerate}
\end{proof}


\section{Proofs from \cref{sec:det}} \label{app:det}

\thmdet*
\begin{proof}
In a Markov chain, one can compute the stationary distribution and hitting probabilities in polynomial time by solving a linear system of equations.
Thus, the numbers $\ell(C)$ defined before \cref{lem:polyprop} can be computed in polynomial time.
Both parts of the theorem follow then from \cref{lem:polyprop}.
A slight complication is that for part~2, for an $\ell = \sum_i x_i \ln y_i \in \Lambda_{\pi_1,\pi_2}$, in order to compute $\PP_{\pi_2}(E_\ell)$ we have to sum the hitting probabilities for all $C$ with $\ell = \ell(C)$.
To select those~$C$ we have to compare numbers of the form $\sum_i x_i \ln y_i$ where $x_i,y_i \in \QQ$, and it is not immediately obvious how to do that.
However, one can compare two such numbers for equality in polynomial time as shown in~\cite{EtessamiSY14}.
\end{proof}




















\section{Proofs from \Cref{sec:rep}}

\fromgentonongen*

\begin{proof}
Since for each $p \in Q_2$, $\sum_{a \in \Sigma} \sum_{q \in Q_2} \Psi_2(a)_{p,q} = 1$ it follows that we may define a function $\kappa_p : [0,1) \rightarrow Q_2 \times \Sigma$ such that for all $p \in Q_2$, The Lebesgue measure $\MLeb(\kappa_p^{-1}\{(q, a)\}) = \Psi(a)_{p,q}$.
Let $\Sigma'$ be the set of atomic elements of the finite $\sigma$-algebra $\sigma\{\kappa_p^{-1}\{(q, a)\} \mid p,q \in Q_2, a \in \Sigma\}$. Let $Q' = C$. We also define the transition matrix $\Psi : \Sigma' \rightarrow \RR_{\geq 0}^{C \times C}$
\begin{equation*}
\Phi(a')_{(q_1,q_2),(r_1,r_2)} = \begin{cases}
\Psi(a)_{q_1, r_1} & (r_2, a) \in \kappa_{q_2}^{-1}(a') \\
0 & \text{else}. \\
\end{cases}
\end{equation*} 
The triple $\mathcal{M} = (Q', \Sigma', \Phi)$ is a matrix system with a strongly connected graph due to $C$ being a bottom SCC. The pair $\mathcal{S'} = (\mathcal{M}, \MLeb)$ is a Lyapunov system (a representation of) which can be computed in $O(|Q_2|^2|Q_1|^2 |\Sigma|)$ time. 

Given a starting state $r_0$ we may extend the mapping $\kappa_{r_0} : [0,1)^n \rightarrow (Q_2 \times \Sigma)^n$ for $n \in \NN$ by letting $\kappa_{r_0}(a'w) = (a, r_1) \kappa_{r_1}(w)$ where $\kappa_{r_0}(a') = (a, r_1)$. Fix $(r_1, a_1) \cdots (r_n, a_n) \in (Q^2 \times \Sigma)^n$ and let $C^k = \{q_1 \cdots q_n \in Q_1^n \mid (q_1, r_1), \dots, (q_k, r_k) \in C\}$ for $k \leq n$. For a word $w' = a_1' \cdots a_n' \in \{\kappa_{j_0}(a_1' \dots a_n') = (r_1, a_1), \cdots, (r_n, a_n)\}$ and start and end states $(q_0, r_0)$ and $(q_n, r_n)$ respectively, we have
\begin{align*}
\Phi(w')_{(q_0, r_0), (q_n, r_n)} & = \sum_{(q_1, \dots, q_{n-1}) \in C^1} \Phi(a_1')_{(q_0, r_0),(q_1,r_1)} \dots \Phi(a_n')_{(q_{n - 1}, r_{n - 1}),(q_n,r_n)}\\
& = \sum_{(q_1, \dots, q_{n-1}) \in C^{n-1}} \Psi(a_1)_{q_0, q_1} \dots \Psi(a_n)_{q_{n - 1}, q_n}\\
& = \Psi(a_1 \cdots a_n)_{q_0, q_n}.
\end{align*}
Further, we have
\begin{align*}
\Psi(a_1)_{r_0, r_1} \dots \Psi(a_n)_{r_{n - 1}, r_n} & = \sum_{a_1' \cdots a_n' \in [0,1)^n} \MLeb(a_1' \times \dots \times a_n')\1_{\kappa_{r_0}(a_1' \dots a_n') = (r_1, a_1), \cdots, (r_n, a_n)}\\
& = \PP_{\MLeb}(\kappa_{r_0}(a_1' \dots a_n') = a_1 \dots a_n).
\end{align*}
Write $\pi_1 \times \pi_2$ for the distribution on $C$ given by $(\pi_1 \times \pi_2)_{q,r} = (\pi_1)_q (\pi_2)_r$. It follows from the above that for any measurable set $A \subseteq \RR$, we have
\begin{equation*}
\PP_{\pi_2}(\pi_1 \Psi(a_1\cdots a_n) \in A) = \PP_{\MLeb}(\pi_1 \times \pi_2 \Phi(a_1'\cdots a_n') \in A)
\end{equation*}
which implies both points in the lemma.
\end{proof}

\constructLsystems*

\begin{proof}
We define the following sets
\begin{align*}
C^* & = \{(q,r) \in Q \times Q \mid \supp (\pi_1) \times \supp (\pi_2) \rightarrow_{G_{\mathcal{H}, \mathcal{H}}} \{(q, r)\}\} \\
C_\perp & = \{(q,r) \in Q \times Q \mid (q,r) \text{ has no outgoing transitions in }G_{\mathcal{H}, \mathcal{H}} \} \\
C_\text{b} & = \{(q,r) \in Q \times Q \mid (q,r) \text{ is in a bottom SCC of }G_{\mathcal{H}, \mathcal{H}} \} \\
C_0 & = C^* \cap (C_\perp \cup C_{\text{bottom}})
\end{align*}
Conditioned on a produced word $r_0 a_1 r_1 \cdots a_m r_m \in (\supp~\pi_1)(\Sigma Q)^*$ we have
\begin{align*}
\lim_{n \rightarrow \infty}\frac1n \ln\frac{\|\pi_1 \Psi(a_1 \cdots a_m) \Psi(w_n)\|}{\|\pi_2 \Psi(a_1 \cdots a_m) \Psi(w_n)\|} & = \lim_{n \rightarrow \infty}\frac1n \ln\frac{\|\pi_1 \Psi(a_1 \cdots a_m) \Psi(w_n)\|}{\|\delta_{r_m} \Psi(w_n)\|} \\
& - \lim_{n \rightarrow \infty}\frac1n \ln\frac{\|\pi_2 \Psi(a_1 \cdots a_m) \Psi(w_n)\|}{\|\delta_{r_m} \Psi(w_n)\|} \\
& = \max_{q \in \supp~ \pi_1 \Psi(a_1 \cdots a_m)}\lim_{n \rightarrow \infty}\frac1n \ln\frac{\|\delta_q \Psi(w_n)\|}{\|\delta_{r_m} \Psi(w_n)\|}.
\end{align*}
Hence, we have $\Lambda_{\pi_1,\pi_2} \subset \{\ell \in \Lambda_{\delta_q, \delta_r} \mid (q, r) \in C^*\}$. Further for any produced word, $r_m$ is almost surely in a bottom SCC of $\sum_{a \in \Sigma}\Psi(a)$.

We prove the claim
\begin{equation}
\{\ell \in \Lambda_{\delta_q, \delta_r} \mid (q, r) \in C\} \subseteq \{-\infty\} \cup \{\lambda(\mathcal{S}^1_R) - \lambda(\mathcal{S}^2_R) \mid R \in \mathcal{R}, R \subseteq C^*\}
\end{equation}\label{indclaimsccs}
for any $C \subseteq C^*$ such that there is no path in $G_{\mathcal{H}, \mathcal{H}}$ from $C$ to $C^* / C$. Consider the case $(q, r) \in C_\text{b}$ then 
\begin{equation*}
\Lambda_{\delta_q, \delta_r} \subseteq \{-\infty, \lambda(S^1_R) - \lambda(S^2_R)\}
\end{equation*}
In the case that $(q, r) \in C_\perp$ then $\Lambda_{\delta_q, \delta_r}= \{-\infty\}$. Hence, the claim holds for $C_0 \subseteq C^*$. Clearly there can be no path from $C_0$ to $C^* / C_0$ since $C_0$ consists of bottom SCCs and states with no outgoing transitions.

We proceed by induction on the DAG structure of the SCCs in $C^*$. 

Assume the claim given in \Cref{indclaimsccs} holds for some $C \subset C^*$ such that there is no path from $C$ to $C^* / C$. Then we first consider the set $\{(q,r)\} \cup C$ where $(q,r) \in C^* / (C_\perp \cup C_\text{b})$ and is transitive with edges only to states in $C$. In this case, the support of $(q, r)$ vanishes after any letter. Hence $\Lambda_{\delta_q, \delta_r} \subseteq \{\ell \in \Lambda_{\delta_{q'}, \delta_{r'}} \mid (q', r') \in C\}$.

We consider now a right-bottom SCC $R \in \mathcal{R}$ such that all outgoing transitions of $R$ lead to $C$. Such an SCC exists because of the DAG structure of $C^*$. We consider the block matrix system $(C \cup R, \Sigma \times Q_2, \overline{\Psi}_{|C \cup R})$. 

For a matrix $M$ let $\|M\|$ be the sum of all its entries. The transition matrix $\overline{\Psi}_{|C \cup R}$ has the form
\begin{equation*}
\overline{\Psi}_{|C \cup R}(w) = \begin{pmatrix}
\overline{\Psi}_{|R}(w) && T \\
0 && \overline{\Psi}_{|C}(w) \\
\end{pmatrix}
\end{equation*}
for any $w = (a_1, r_1)\cdots (a_n, r_n)\in (\Sigma Q_2)^n$ where $T : (\Sigma \times Q_2)^* \rightarrow \RR_{\geq 0}^{R \times C}$ is some matrix valued function with the property
\begin{align*}
\| T \| & \leq \sum_{t = 1}^n \| \overline{\Psi}_{|R}\big((a_1, r_1)\cdots (a_{t-1}, r_{t-1})\big)\| \|\overline{\Psi}_{|C}\big((a_{t+1}, r_{t+1})\cdots (a_n, r_n)\big) \|\\
& + \| \overline{\Psi}_{|R}\big((a_1, r_1)\cdots (a_n, r_n)\big)\| + \| \overline{\Psi}_{|C}\big((a_1, r_1)\cdots (a_n, r_n)\big)\|.
\end{align*} 
In the case that $\lambda(\mathcal{S}^1_R) - \lambda(\mathcal{S}^2_R) > \sup \{\ell \in \Lambda_{\delta_q, \delta_r} \mid (q, r) \in C\}$ then for any $\epsilon > 0$ there is a number $n_\epsilon$ such that
\begin{equation*}
\|\overline{\Psi}_{|C}\big((a_{n_\epsilon+1}, r_{n_\epsilon+1})\cdots (a_n, r_n)\big) \| \leq \|\overline{\Psi}_{|R}\big((a_{n_\epsilon+1}, r_{n_\epsilon+1})\cdots (a_n, r_n)\big) \|
\end{equation*}
with $\PP_{\delta_r}$-probability at least $1 - \epsilon$. Hence, 
\begin{equation*}
\lim_{n \rightarrow \infty} \frac1n \ln \|\overline{\Psi}_{|C \cup R}(w_n)\| \in \{\lambda(\mathcal{S}^1_R) - \lambda(\mathcal{S}^2_R) \} \cup \{-\infty\} \cup \{\lambda(\mathcal{S}^1_{R'}) - \lambda(\mathcal{S}^2_{R'}) \mid R' \in \mathcal{R}, R' \subseteq C\}.
\end{equation*}
Thus the claim holds for $C \cup R$. By induction the claim then holds for $C = C^*$ which completes the proof.
\end{proof}